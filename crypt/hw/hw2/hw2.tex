\documentclass[10pt]{article}        
\newcommand{\suchthat}{\;\ifnum\currentgrouptype=16 \middle\fi|\;}
\usepackage{amsmath,amsfonts,amssymb,enumerate}   
\title{Hw 1}
\author{Matthew Yates}
\begin{document}

\maketitle

\begin{abstract}
	This is my homework. I copy and paste text from the eBook and latex is easier to use
	with raw text symbols so I am going to use Latex.
\end{abstract}

\section{Breaking a Vigenere Cipher}
\subsection{Cipher Text}

VGKGGVQINMVNGEITBFOVWBBBEMCBBN
YPUBGYVOATZVGXHXSZCXUTHGVGWGRL
BYDTJRGSOMCPXMPPWGUWQVVSHVJBHH
EIWLMBHGDHCFREBHCYGEGXGVAEHEHU
RPQFPRESBXJRECOHFGGLWMVRPSIXGN
PVKLGFBXDTHGUIGGCJYIZZSJUMYAAV
TLPUSHFIBNZGBLEFURGWYKCJQIZHIG
BVWMPRFXELXHZFHXRHCAEMVNYSPHTB
GLAKHUVRCLGBGLWMVRUEOTRVSJEVIY
GCEGZNLMJZVVFLWGRFHTKGWGASSMVR
FOEETHYAKKYZNRELJRECYTFRSYHBBQ
RIZTGGBADTHURXWDSFVRPHVVFFNTWA
UISBZYUERXBBGLEGUOHXPASGBSHLKU
VGDFOLUIHIVVZMJWCVAKDBGJBVGUIG
BJPASFRLAAOFNPWKURNWOHFGZIJMOA
QEHEWAGLAFCFGTAKTRPXKKRREMPBGN
ZMOMOXRXKMVVAOPAOGGLWMZVGXHXFB
BQDTGRYEOMWPJEHEGNAHYTBQVWPXBQ
GSWGMRKXAGHQRTAGRHCSJBHGUINXQB
ZIOTHVZISASASSNXJRECWWRVGMKGCS
XRKPZRQKARCHSSNZSGFSIXHUVRCMVN
GCKNYARAXXTBEIEMWFBJPASUVKDXGG
VQLHFGNRYXHURVAYCERRKMHBUERXIF
RPALGSNGPLSYOSSBBTBYPMVRHWAYIY

\subsection{Answer}
I consider that a man's brain originally is like a little empty attic, 
and you have to stock it with such furniture as you choose. A fool takes 
in all the lumber of every sort that he comes across, so that the knowledge
which might be useful to him gets crowded out, or at best is jumbled
up with a lot of other things, so that he has a difficulty in laying his 
hands upon it. Now the skillful workman is very careful indeed as to what
he takes into his brain-attic. He will have nothing but the tools which
may help him in doing his work, but of these he has a large assortment,
and all in the most perfect order. It is a mistake to think that that
little room has elastic walls and can distend to any extent. Depend 
upon it there comes a time when for every addition of knowledge you forget 
something that you knew before. It is of the highest importance, 
therefore, not to have useless facts elbowing out the useful ones.

-- Arthur Conan Doyle, A Study in Scarlet

\section{Chapter 1 Problem 28}
\subsection{Question}
Decrypt the following ciphertext, obtained from the Autokey Cipher, by using exhaustive
key search.

\subsection{Cipher Text}
MALVVMAFBHBUQPTSOXALTGVWWRG

\subsection{Answer}


\section{Problem 3.3}
\subsection{Question from book}
Let DES(x, K) represent the encryption of plaintext x with key K using the DES
cryptosystem. Suppose $y = DES (x, K)$ and $y^\prime  = DES (c(x), c(K))$, where $c(-)$ denotes the bitwise
complement of its argument. Prove that $y^\prime  = c(y)$ (i.e., if we complement the plaintext and the
key, then the ciphertext is also complemented). Note that this can be proved using only the “high-
level” description of DES — the actual structure of S-boxes and other components of the system
are irrelevant.

\subsection{Answer}




\section{Problem based off 1.29}
\subsection{Explanation and Problem from the book}
We describe a stream cipher that is a modification of the Vigenere Cipher. Given a keyword
$(K_{1}, . . . , K_{m})$ of length m, construct a keystream by the rule 
$z_{i} = K_{i} (1 ≤ i ≤ m), z_{i} + m = z_{i} + 1 mod 26 (i \geq m + 1)$. In other words, each time we use the keyword, we replace each letter by its
successor modulo 26. For example, if SUMMER is the keyword, we use SUMMER to encrypt the
first six letters, we use TVNNFS for the next six letters, and so on.
Describe how you can use the concept of index of coincidence to first determine the length of the
keyword, and then actually find the keyword.
\subsection{Cipher Text}
DERMSPQGGFBCLAFZZQXICUBJWNPACC
PFXNGPHRBMWQSJIWWAFFNLXEUVTNJD
UXHDYTOEUYQEXMEOPYQPYGUQJDHOSC
YYRZDVFWBZCNTUTJEGIFLHQUBDDHYT
RYISHNLTYFMERLVEWSMBDSWBCVPXPO
ZDZYJDUERIJDPPYTICEHQQLCZIDEQU
NKQVCTZVTPFZHKXUTMUJAPMITNDXJZ
AGUQUKOQDCBLVKWWONCLGCWMLUGNNG
YIQHSSITLALNPHUQOYVYCIUMHNDFBH
RDNLILFMCMNYOFASUWCLXGLFRJTDPF

\subsection{Answer}


\end{document}

