\documentclass[10pt]{article}        
\newcommand{\suchthat}{\;\ifnum\currentgrouptype=16 \middle\fi|\;}
\usepackage{amsmath,amsfonts,amssymb,enumerate}   
\title{Hw 1}
\author{Matthew Yates}
\begin{document}

\maketitle

\begin{abstract}
	This is my homework. I got tired of using liber office 
	to write logical symbols so I am going to use Latex.
\end{abstract}

\section{Chapter 1 Problem 2}
\subsection{Question}
In considering the distinction between knowledge and belief in this book,
we take the view that belief is fundamental, and that knowledge is simply
belief where the outside world happens to be cooperating (the belief is true,
is arrived at by appropriate means, is held for the right reasons, and so on).
Describe an interpretation of the terms where knowledge is taken to be basic,
and belief is understood in terms of it.

\subsection{Answer}
Belief is that $E=Mc^{2}$. Knowledge is proof that through
general relativity that mass multiplied speed of light 
squared is the amount of energy gained if an amount
of matter is annihilated. 

Knowledge is how all of the various concepts are linked together.

\section{Chapter 1 Problem 4}
\subsection{Question}
It has become fashionable to attempt to achieve intelligent behavior in AI
systems without using propositional representations. Speculate on what such
a system should do when reading a book on South American geography.

\subsection{Answer}
While reading a book on South American geography the AI would start to build
a knowledge base, by writing rules linking mountains to Andes and such.
By the end there would be many rules linking many other topics.

\section{Chapter 1 Problem 5}
\subsection{Question}
Describe some ways in which the first-hand knowledge we have of some
topic goes beyond what we are able to write down in a language.
What accounts for our inability to express this knowledge?

\subsection{Answer}
It is impossible to completely define some very simple concepts
such as joy and sadness with out resulting in using synonyms.
It is because people apply difficult to define emotions to simple words.

\section{Chapter 2 Problem 1}
\subsection{Question}
For each of the following sentences, give a logical interpretation that makes
that sentence false and the other two sentences true:

\begin{enumerate}
	\item $ \forall x $ and $\forall y $ and $\forall z \suchthat  P(x,y) \land P(y,z) \models P(x,z) $
	\item $ \forall x $ and $\forall y \suchthat  P(x,y) \land P(y,x)  \models (x = y) $
	\item $ \forall x $ and $\forall y \suchthat  P(a,y) \models P(x,b) $
\end{enumerate}

\subsection{Answer}

\begin{enumerate}
	\item $ P(q,p) {(q \equiv a) \lor (p \equiv b) \lor (p \equiv True) } $
	\item $ P(q,p) {True} $
	\item $ P(q,p) {q \models p }  $
\end{enumerate}

\section{Chapter 2 Problem 4}
\subsection{Question}
In a certain town, there are the following regulations
concerning the town barber:

Anyone who does not shave himself must be shaved
by the barber.

Whomever the barber shaves, must not shave himself.

Show that no barber can fulfill these requirements. That is, formulate the
requirements as sentences of FOL, and show that in any interpretation where
the first regulation is true, the second one must be false.
(This is called the barber’s paradox and is due to Bertrand Russell.)

\subsection{Answer}

\begin{enumerate}
	\item $\forall x \in Town-Members \suchthat \lnot shavedBy(x,x) \to shavedBy(x,Barber) $
	\item $\forall x \in Town-Members \suchthat shavedBy(x,Barber) \to \lnot shavedBy(x, x) $
\end{enumerate}

But Town Member x is the Barber both of the statements become a paradox

\begin{enumerate}  
	\item $ \lnot(shavedBy(Barber,Barber)) \to shavedBy(Barber,Barber)$
	\item $ shavedBy(Barber,Barber) \to \lnot shavedBy(Barber,Barber)$
\end{enumerate} 

\end{document}

